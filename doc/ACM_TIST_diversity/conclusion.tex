In this article, we made fundamental contributions to the theoretical
understanding of expected $n$-call@$k$ (guided by simpler 
derivations for $n=1,2$) and its relationship with
diversity.  In contrast to a variety of diverse retrieval algorithms
derived from alternate rank-based relevance criteria such as average
precision and reciprocal rank, we provided a complementary theoretical
perspective on the emergence of diversity via optimization of
$n$-call@$k$ in a latent subtopic model of relevance that focused on
the subtopic motivation for diverse retrieval.  Further, we
precisely formalized the mathematical relationship between
$n$-call@$k$ and diversity and confirmed empirical observations in the
literature.  Altogether, these theoretical results have provided deep
connections to the popular MMR diversification algorithm and both
motivated and contrasted with other diversification
approaches, hence providing a new theoretical basis for the
investigation of diverse retrieval algorithms.

%% SHENGBO:
%%
%% This future work does not follow on clearly from the analysis in this paper...
%% it's not a good note to end on.  If you're going to include these citations,
%% you need to find some clear way to put them in the related work section.
%Future work includes the study of efficiently optimizing the
%n-call@$k$ objecive function by possibly exploiting the
%submodularity \cite{Borodin:Database2012}. Additional, it is also
%important to study the temporal diveristy as suggested
%in \cite{lathia10TemporalDiversity} and \cite{Zhao:SIGIR2012}.

\appendix
