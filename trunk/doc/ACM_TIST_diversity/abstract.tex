In this article, we explore the relationship between result set
diversification and the optimization of \mbox{$n$-call@$k$} -- a
set-based relevance objective that is 1 if at least $n$ documents in a
set of $k$ are relevant, otherwise 0.  First, we formally quantify the
mathematical relationship between diversity and greedy optimization of
the \emph{expected} $n$-call@$k$ objective in a latent subtopic model
of binary relevance.  Second, we relate this result to a variety of
other diversification approaches proposed in the literature, including
deep connections with maximal marginal relevance.  The contributions
of this work are threefold: (1) in contrast to a variety of diverse
retrieval algorithms derived from alternate rank-based relevance
criteria such as average precision and reciprocal rank, we provide a
complementary theoretical perspective on the emergence of diversity
via optimization of $n$-call@$k$ in a latent subtopic model of
relevance intended to model both ambiguous and faceted subtopic
retrieval, (2) we precisely formalize a mathematical relationship
between $n$-call@$k$ and diversity that confirms empirical
observations in the literature, and (3) we provide a theoretical
underpinning and comparison to many other (ad-hoc) diversification
approaches in the literature.



