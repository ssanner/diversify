One of the basic tenets of set-based information retrieval is to
minimize redundancy, hence maximize diversity, in the result set to
increase the chance that the results will contain items relevant to
the user's query~\cite{goffman64OnRelevanceAsAMeasure}.  Hence, \emph{diverse retrieval}
can be defined as a \emph{set-level} retrieval objective that takes
into account inter-document relevance dependences when producing a
result set relevant to a query.

\emph{Subtopic retrieval} --- ``the task of finding documents that
cover as many \emph{different} subtopics of a general topic as
possible''~\cite{zhai03Beyond} --- is a
motivating case for diverse retrieval.  That is, if a query has
multiple facets that should be covered by a result set, or a query has
multiple ambiguous interpretations, then a retrieval algorithm should
try to ``cover'' all of these subtopics in its result set.

One of the most popular result set diversification methods is Maximal
Marginal Relevance (MMR)~\cite{carbonell98MMR}.  
Formally, given an \emph{item set} $D$ (e.g., a set of documents) where retrieved items
are denoted as $s_i \in D$, we aim to select an optimal subset of
items $S_k^* \subset D$ (where $|S_k^*| = k$ and $k < |D|$)
\emph{relevant} to a given query $\vec{q}$ (e.g., query terms) with
some level of \emph{diversity} among the items in $S_k^*$.  MMR
builds $S_k^*$ in a greedy manner by choosing the next optimal
selection $s_k^*$ given the set of $k-1$ optimal selections
$S_{k-1}^* = \{ s_1^*, \ldots, s_{k-1}^* \}$ (recursively defining
$S_k^* = S_{k-1}^* \cup \{ s_k^* \}$ with $S_0^* = \emptyset$)
as follows:
%MMR chooses $s_k^*$
%greedily according to the following criteria:
\begin{equation}\label{eq:MMR}
 s_k^* = \hspace{-.3mm} \argmax_{s_k \in D \setminus S_{k-1}^*} [ \lambda(\Sim_{1}(\vec{q},s_k))\hspace{-.3mm}-\hspace{-.3mm}(1-\lambda)\max_{s_i \in S_{k-1}^*} \Sim_{2}(s_i,s_k) ].
\end{equation}
Here, 
$\lambda \in [0, 1]$, metric $\Sim_{1}$ measures
query-item relevance, and metric $\Sim_{2}$ measures the similarity
between two items. In the case of $s_1^*$, the $\max$ term is omitted.

In MMR, we note that the $\lambda$ term explicitly controls the
trade-off between relevance and diversity. This $\lambda$ term has been traditionally set in an ad-hoc manner or
in recent work, learned in a query-specific way from data~\cite{santos2010selectively}.  

Presently, little is formally known about how a particular selection
of $\lambda$ relates to the overall \emph{set-based relevance objective}
being optimized.  However, it has been previously noted that the
$n$-call@$k$ set-based relevance metric (which is 1 if at least $n$
documents in a set of $k$ are relevant, otherwise 0) encourages
diversity as $n \to 1$~\cite{chen06Less,wang09PortfolioTheory}.
Indeed, Sanner \emph{et al.} \cite{sanner11}~have shown that optimizing
\emph{expected $n$-call@$k$} for $n=1$ corresponds to 
$\lambda = 0.5$ --- we extend
this derivation to show that $\lambda = \frac{n}{n+1}$ 
for arbitrary $n \geq 1$ 
(independent of result set size $k \geq
n$).  This result precisely formalizes a relationship
between $n$-call@$k$ and the relevance vs. diversity
trade-off.
